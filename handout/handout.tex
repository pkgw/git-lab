\documentclass[letterpaper,12pt,titlepage]{article}
\input setup

\title{A Laboratory Introduction to \git}
\author{P. K. G. Williams (\href{mailto:peter@newton.cx}{peter@newton.cx})}
\date{\today}

\begin{document}
\maketitle

\section*{Introduction}

Welcome to the \git\ lab! This manual aims to help you learn the fundamentals
of this awesome tool by walking you through some exercises that demonstrate
its everyday functionality. Along the way we'll discuss some of its underlying
principles and demonstrate some basic Unix shell skills too.

\i{This lab is best done with a partner.} Real-deal education research shows
that it's much faster for you to learn something when talking it over with a
partner rather than just staring at it by yourself. If you and your partner
get stuck on something, trying asking the group next to you, or Google. Your
lab assistants are happy to help you out, of course, but you'll learn more if
you spend some time trying to solve problems on your own. That being said,
with computers sometimes things go wrong even if you've done everything right
--- if you see a truly strange error message, particularly one associated with
a non-\git\ command, it's probably best to summon the lab assistants sooner
rather than later.



Notation: important terminology will be introduced in \i{italics}. Computer-y
words will be written in a monospace font \t{like this}. Commands that you
should type in are presented this way:

\begin{typeme}
echo hello world & Say hello.
\end{typeme}

\noindent (You shouldn't type the leading dollar sign, which is just meant to
indicate a generic terminal prompt.) Along the way we'll introduce a few Unix
commands that you may not be familiar with; if you'd like to learn more about
one, you can try reading its manual page:

\begin{typeme}
man ls & Learn about \t{ls} \\
man git & Learn about \t{git} \\
man man & Learn about \t{man}
\end{typeme}

\ldots\ although Unix manual pages are notoriously uneven in their
helpfulness. (This is especially true regarding \git\ itself, unfortunately.)


\segment{\textsf{git clone}}

The first task is to set up a \i{git repository} to play with. This is the
directory containing your actual content (\ie, files) as well as \git's
supporting data, which are stored in a hidden subdirectory called
``\t{.git}''. There are two commands to set up a repository: \t{git init},
which creates a new, empty repository; and \t{git clone}, which duplicates an
existing one. We'll use the latter so that we have some files to work with
right off the bat.

\begin{typeme}
cd & Go to home directory. \\
mkdir gitlab & Create work directory. \\
cd gitlab & Move into it. \\
git clone https://github.com/pkgw/bloomdemo.git & Clone an existing repository.
\end{typeme}

\end{document}
