\documentclass[letterpaper,12pt,titlepage]{article}
\input setup

\title{A Laboratory Introduction to \git}
\author{P. K. G. Williams (\href{mailto:peter@newton.cx}{peter@newton.cx})}
\date{\today}

\begin{document}
\maketitle

\section*{Introduction}

Welcome to the \git\ lab! This manual aims to help you learn the fundamentals
of this awesome tool by walking you through some exercises that demonstrate
its everyday functionality. Along the way we'll discuss some of its underlying
principles and demonstrate some basic Linux shell skills too. We should note
that we're assuming that you've already been given a broad overview of what
\git\ is, and why you might want to use it.

Learning \git\ is like learning an instrument, or a language. There are
certain concepts to master, but there's also just a lot of practice involved:
repeating certain motions so that they become automatic. When it comes to
computers, Linux pros talk about \i{finger memory}: I can type out \t{git
  commit -am} in my sleep. Of course, there's also the deeper appreciation of
the underlying concepts and esoteric possibilities that you gain as you use a
tool more and more. My assertion to you is that \i{time spent practicing
  \git\ will more than repay itself in the future}. If you're just starting
out as a programmer (\ie, you've written less than 100,000 lines of code or
so), some of its advantages won't be immediately obvious; for what it's worth,
I've been doing this for 20 years, and if \git\ disappeared tomorrow the first
thing I'd do is start recreating it. The developer site
\href{https://github.com/}{GitHub} has more than 4.5 million users and 6
million \git\ repositories.

\i{This lab is best done with a partner.} Real-deal education research shows
that it's much faster for you to learn something when talking it over with a
partner rather than just staring at it by yourself. Getting back to the finger
memory thing, though, please try to switch off between who's ``driving'' the
keyboard and who's watching and commenting. It really does make a difference!

If you and your partner get stuck on something, trying asking the group next
to you, or Google. Your lab assistants are happy to help you out, of course,
but you'll learn more if you spend some time trying to solve problems on your
own. That being said, with computers sometimes things go wrong even if you've
done everything right --- if you see a truly strange error message,
particularly one associated with a non-\git\ command, it's probably better to
summon the lab assistants sooner rather than later.

Notation: important terminology will be introduced in \i{italics}. Computer-y
words will be written in a monospace font \t{like this}. Commands that you
should type in are presented this way:

\begin{typeme}
echo hello world & Say hello.
\end{typeme}

\noindent You shouldn't type the leading dollar sign, which is just meant to
indicate a generic terminal prompt. Along the way we'll introduce a few Unix
commands that you may not be familiar with; if you'd like to learn more about
one, you can try reading its manual page:

\begin{typeme}
man ls & Learn about \t{ls} \\
man git & Learn about \t{git} \\
man man & Learn about \t{man}
\end{typeme}

\ldots\ although Unix manual pages are notoriously uneven in their
helpfulness. (This is especially true regarding \git\ itself, unfortunately.)
Google is often a better source of information for beginners. The information
on the \href{http://stackexchange.com/}{StackExchange.com} websites is often
particularly relevant.

Well, that's all the preliminaries. Let's get started!


\segment{\textsf{git clone}}

The first task is to set up a \i{git repository} to play with. This is the
directory containing your actual content (\ie, files) as well as \git's
supporting data, which are stored in a hidden subdirectory called
``\t{.git}''. There are two commands to set up a repository: \t{git init},
which creates a new, empty repository; and \t{git clone}, which duplicates an
existing one. We'll use the latter so that we have some files to work with
right off the bat.

\begin{typeme}
cd & Go to home directory. \\
mkdir gitlab & Create work directory. \\
cd gitlab & Move into it. \\
git clone https://github.com/pkgw/bloomdemo.git & Clone an existing repository.
\end{typeme}

\end{document}
