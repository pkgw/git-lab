% Page setup. We use a landscape letter layout, since we want lots of room for
% marginal notes since I want to provide commentary without overloading people
% with too much to read. Margins are 1" except for 0.5" at the top, which gets
% augmented with a nice header. The marginal paragraphs are 2.25" with a 0.25"
% gutter, leaving 6.5" for the main body text.
\usepackage[
  letterpaper,
  landscape,
  includehead,
  includemp,
  margin = 1in,
  marginparwidth = 2.25in,
  marginparsep = 18pt,
  top = 0.5in,
  headheight = 15pt
]{geometry}

% Headers and footers.
\usepackage{fancyhdr}
\pagestyle{fancyplain}
\fancyhf{}
\fancyhead[LE,RO]{\thepage}
\fancyhead[LO,RE]{\textsf{\leftmark}}
\renewcommand\sectionmark[1]{\markboth{#1}{}} % needed for good headers.

% Fonts. Note that EB Garamond has no bold, alas!
\usepackage{fontspec}
\setmainfont{EBGaramond12}[
  Extension = .otf,
  UprightFont = *-Regular,
  ItalicFont = *-Italic
]
\setmonofont{SourceCodePro}[
  Extension = .otf,
  UprightFont = *-Regular,
  BoldFont = *-Bold,
  ItalicFont = *-It,
  BoldItalicFont = *-BoldIt,
  Scale = MatchLowercase
]

% Links
\usepackage[
  breaklinks,
  colorlinks,
  urlcolor=blue,
  citecolor=blue,
  linkcolor=blue
]{hyperref}

% Skips between pars rather than indents at their beginnings; works better for
% this kind of document, I think:
\usepackage[parfill]{parskip}

% Marginal notes. We'll use a lot of them.
\usepackage{marginnote}
\renewcommand*{\marginfont}{\small}

% To get the right vertical positioning of the marginal notes, you need to put
% the \marginnote command in the right place, but this messes up the textual
% flow of paragraphs in the TeX file pretty badly. Here's a dumb
% hack: \xnote{stuff} is an abbreviation for \def\x{\marginnote{stuff}}. Use
% this command before a main-body paragraph, and then just put a little \x
% inside the paragraph text instead of the contents of the full margin note.
\newcommand{\xnote}[1]{\def\x{\marginnote{#1}}}

% Sectioning.
\usepackage{titlesec}
\titleformat{\section} {\sffamily\Large\bfseries}{Topic \thesection:}{1ex}{}
\titleformat{\subsection} {\sffamily\large\bfseries}{\thesubsection}{1em}{}
\titleformat{\subsubsection} {\sffamily\normalsize\bfseries}{\thesubsubsection}{1em}{}
\titlespacing*{\section} {0pt}{2.5ex plus 1ex minus 0.2ex}{1ex plus 0.2ex}
\titlespacing*{\subsection} {0pt}{2ex plus 0.8ex minus 0.2ex}{1ex plus 0.1ex}
\titlespacing*{\subsubsection} {0pt}{3.25ex plus 1ex minus 0.2ex}{1.5ex plus 0.2ex}

% Title formatting.
\usepackage{titling}
\pretitle{\begin{center}\Huge}
\posttitle{\end{center}}

% Miscellaneous small macros.
\let\b=\textbf
\let\i=\textit
\let\s=\textsf
\let\t=\texttt
\newcommand\cmd[1]{\s{git~#1}}
\newcommand\git{\s{git}}
\newcommand\github{\s{GitHub}}
\newcommand\ie{\i{i.e.}}

% The question-to-answer environment. Derived from `leftbar` environment of
% `framed` package. We just make the rule thinner.
\usepackage{framed}

\newcommand\fillme[2]{%
  \bigskip%
  \begingroup%
  \def\FrameCommand{\vrule width 1pt \hspace{10pt}}%
  \MakeFramed {\advance\hsize-\width \FrameRestore}%
  \i{#2}\rule[-#1]{0in}{#1}%
  \endMakeFramed%
  \endgroup%
}

% The commands-to-type environment. Wow. Much magic. The table is set to 6.5"
% width, matching the main body text size.
\usepackage{tabularx}
\usepackage[table]{xcolor}

\newenvironment{typeme}{%
  \tabularx{6.5in}{>{\columncolor{lightgray}\ttfamily\i{\$} }X>{\sffamily}l}%
}{%
  \endtabularx%
}

% Shorthand for a one-line command-to-type.
\newcommand\typeone[2]{%
  \begin{typeme}%
  #1 & #2%
  \end{typeme}%
}

\newcommand\fillinparam[1]{\s{\b{\{#1\}}}}
\let\p=\fillinparam

% Boxes around sample characters in the Introduction.
\setlength{\fboxsep}{1pt}

% Embedded versions
\newcommand\demohead{90984a} % NOTE! sync with bloomdemo.
\input{gitversion} % definition generated during build process
